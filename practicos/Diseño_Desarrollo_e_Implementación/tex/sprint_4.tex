\section{Sprint 4}%mostrar gráficas -validaciones
\subsection{Planificación}

\textbf{Inicio: }Martes 17 de julio del 2015 

\textbf{Fin:} Martes 11 de Septiembre del 2015

\begin{figure}[h!]
  \centering
  \includegraphics[width=.8\textwidth]{img/4-PR_back}
  \caption{Pull request realizados por el back end en el sprint  34}
  \label{4-PR_back}
\end{figure}
\begin{figure}[h!]
  \centering
  \includegraphics[width=.8\textwidth]{img/4-PR_front}
  \caption{Pull request realizados por el front end en el sprint  3}
  \label{4-PR_front}
\end{figure}

	{\scriptsize
	\begin{center} %sidewaystable
	\centering
	%\begin{adjustbox}{max width=\textheight}
    \resizebox{\textwidth}{!}
    {
	\begin{tabular}{|l|l|l|p{5cm}|l|p{1cm}|}
		\hline
		\textbf{Area a cargo} &
		\textbf{Responsable} &        
		\textbf{Revisor} &        	        
		\textbf{Tarea} &
		\textbf{US} &
		\textbf{Tiempo dedicado} \\
   		\hline
       
	    Back-end& Franco Canizo& Michael Manganiello & Agrega a la estructura de la API paquete para validadores generales.  & US-\ref{resumenInfo} \& US-\ref{infoSalud}& 8hs\\ \hline
	    Back-end& Franco Canizo& Michael Manganiello & Creación de validadores para números enteros positivos, fecha, fecha-hora, fecha-hora previa, string con\/sin números  & US-\ref{resumenInfo} \& US-\ref{infoSalud} &8hs\\ \hline
	    Back-end& Michael Manganiello& Franco Canizo & filtrado de mediciones en base al tipo, fuente y unidad de medición & US-\ref{resumenInfo} \& US-\ref{infoSalud}&9hs \\ \hline
	    Back-end& Michael Manganiello& Franco Canizo & Corrección error de comparación de fechas con información de zona horaria y fechas sin esa información & US-\ref{resumenInfo} \& US-\ref{infoSalud} &8hs\\ \hline
	    Back-end& Michael Manganiello& Franco Canizo & Simplificación del parseo de argumentos date y datetime & US-\ref{resumenInfo} \& US-\ref{infoSalud} &5hs \\ \hline        
   	    Front-end& Ivan Terreno & & Capacitación en D3 & \textbf{US-17} \& \textbf{US-15}&18hs\\ \hline
   	    Front-end& Ivan Terreno& Michael Manganiello & Generación de gráficas & \textbf{US-\ref{graficaParaPaciente}} \& \textbf{US-15}&18hs\\ \hline        
	    \end{tabular}
        }
	    %\end{adjustbox}
    	\end{center}
	}


\subsection{Descripción}
%%Descripción de Franco de validadores

Además se describirán las tareas realizadas para realizar las gráficas de las distintas mediciones

\subsection{User Stories relacionados}
La \textbf{Tabla \ref{US-Sprint4} } indicará las características de cada user story para guiarnos en el desarrollo del sprint.

\begin{table}[h]
    \label{US-Sprint4}
	%\resizebox{\textwidth}{!}{
    \centering
	\begin{tabular}{|l|p{14cm}|}
	\hline
        \multicolumn{1}{|c|}{\textbf{ID}} &
        \multicolumn{1}{|c|}{\textbf{Enunciado de la historia}} \\          
    \hline
       \ref{infoSalud} &
       Como paciente quiero cargar mi información personal de salud referido a mediciones (altura, grasa corporal, peso, presión arterial), para que el médico cuente con más y mejor información al momento de realizar el diagnóstico. 
    
       
       \\
       \hline
	    \ref{graficaParaPaciente} &
	    Como paciente quiero ver gráficas que resuman mi información en particular para poder ver mis cambios a lo largo de la historia. 

	    
	    \\
	    \hline
	    \ref{resumenInfo} &
	    Como paciente quiero obtener un resumen de mi información de salud básica para hacer uso de la misma en caso de una emergencia. 

	    
	    \\
	   \hline      
        \ref{graficaParaMedico} & Como médico quiero ver gráficas que resuman la información de un paciente para poder ver sus cambios a lo largo de la historia y así apoyar la toma de decisiones y el diagnóstico.\\
    \hline
    \end{tabular}
%     }

\end{table}

\subsection{Modelo funcional} 
Se describirán las funciones usando como marco de apoyo el sprint Backlog,
además se armará el diagrama de casos de uso del presente Sprint [Figura 44] que
irá creciendo medida se vaya avanzando en el proyecto.

\begin{comment}
	Faltaría diagrama de casos de uso para esta parte.
\end{comment}


\begin{comment}
\subsubsection{Creación de página de mediciones}
\end{comment}

\subsubsection{Validadores}
Del lado del backend el trabajo consistió en la definición de validadores usados para controlar errores y, de ser posible, sanear los datos que envían las representaciones a los recursos. Lo que se hizo es definir en un paquete una serie de validadores generales usados en el paquete \textbf{parsers} el cual se encuentra en \textbf{common}. Como podemos apreciar en la definición del parser usado por el recurso de mediciones:

\begin{lstlisting}[language=Python]
parser = reqparse.RequestParser()
parser.add_argument('datetime', type=is_valid_previous_datetime, required=True)
parser.add_argument('value', type=float, required=True)
parser.add_argument('analysis_id', type=is_valid_id, required=True)
parser.add_argument('profile_id', type=is_valid_id, required=True)
parser.add_argument('measurement_source_id', type=is_valid_id)
parser.add_argument('measurement_type_id', type=is_valid_id, required=True)
parser.add_argument('measurement_unit_id', type=is_valid_id, required=True)
\end{lstlisting}

Se define en el argumento type el llamado a una función is\_valid\_previous\_datetime, está función está definida en el paquete validators en el archivo generic validators.

\begin{lstlisting}[language=Python]
    datetime_var = is_valid_datetime(var)
    if datetime_var.year < 1900:
        raise ValueError("La fecha y hora ingresada no puede ser anterior al anio 1900.")
    elif datetime_var > datetime.utcnow():
        raise ValueError("La fecha y hora ingresada no debe ser posterior a la fecha y hora actual.")
    else:
        return datetime_var
\end{lstlisting}

Este a su vez llama al validador is\_valid\_datetime para controlar que el dato ``datetime'' tenga un valor de fecha-hora correcto. Si el validador no encuentra error devuelve la fecha-hora, en caso contrario lanza un error.
En este parser también se define un método para controlar que el id recibido es valido, por el momento lo que hace este es controlar que el id recibido sea positivo.
\begin{comment}
	Se podría poner como cosas a mejorar la validación de id.
\end{comment}

\subsubsection{Filtrado de mediciones}
El backend identifica como útil la definición de un recurso que permita devolver medidas de un perfil específico filtradas según tipo, fuente y unidad de medida. Para esto se agregó un recurso \/profile\/id\/measurements en el archivo principal de la aplicación y un recurso que define un método get para responder a una solicitud HTTP con operador GET. El mismo utiliza los datos recibidos en el ``query string'' de la URL para determinar según que tipo, fuente o unidad de medida debe filtrar. El recurso en sí no se encarga del filtrado sino que toma los datos verificados por el parse, obtiene los datos del query string y pasa estos datos a un método auxiliar definido en el archivo measurement del paquete persistence de la API ``get\_by\_profile'' el cual devuelve todas las instancias existentes de medición, asociadas a un perfil específico, ordenadas por fecha y hora de la medición, y filtradas por fuente, tipo y unidad de medición. Una vez recibida la respuesta arma el response con el código de estado correspondiente y en el cuerpo los datos resultantes serializados de acuerdo a la representación utilizada.



\subsubsection{gráficas}
Para las gráficas se utilizo D3, fue necesario añadir código javascript, el cual fue añadido con grunt
\subsection{Salidas del sistema}
El resultado de la implementación del código necesario para mostrar las gráficas del peso se puede ver en la \textbf{[Figura \ref{5-grafica_medicion}]}, 


\begin{figure}[h!]
  \centering
  \includegraphics[width=.8\textwidth]{img/5-grafica_medicion}
  \caption{vista generada para una medición en particular}
  \label{5-grafica_medicion}
\end{figure}


\begin{comment}
	Faltaría agregar algunas pruebas con curl y algo de código.
    Poner que se desarrollo para pasar datos específicos para que el front-end presente datos resumidos cuando estos son muchos y complejos??
\end{comment}

\subsection{Criterios de aceptación}

\begin{center}
\begin{longtable}{|p{0.5cm}|p{4cm}|p{4cm}|p{5cm}|}
\hline \hline \rowcolor[gray]{0.9}
	\multicolumn{4}{||c|}{\textbf{Criterio de aceptación}} \\
    \hline  \rowcolor[gray]{0.9}
        \textbf{Id} &
        \textbf{Contexto} &
        \textbf{Evento}&
        \textbf{Resultado} \\
    \hline
1&En caso de que se envíe como id un valor menor o igual a 0 en cualquier representación  de una solicitud & Al ejecutar el método post, get, delete o put correspondiente del recurso & El sistema devolverá un json con el mensaje de error por entero no positivo y el código de error 400 \\ \hline
	\hline
2&En caso de que se indique una fecha cuyo año sea menor a 1900 & al ejecutar el validador is\_valid\_previous\_date del argumento  & El sistema devolverá un json con el mensaje de error correspondiente a una fecha previa no permitida y el código de estado 400 \\ 		\hline
	\hline
3&En caso de que se indique en la solicitud una fecha-hora con formato invalida 
& Al ejecutar el validador \textbf{ is\_valid\_datetime}  & El sistema devolverá un json con el mensaje de error correspondiente a una fecha hora con formato inválido y el código de estado 400\\ \hline
    \hline
4&En caso de que no exista un usuario registrado con el id indicado & al ejecutar el método get\/put del recurso \/users\/id  & El sistema devolverá un json con un mensaje de error y un código de error 404 \\ \hline
	\hline
5&En caso de que no exista un perfil registrado con el id indicado & al ejecutar el método get del recurso \/profile\/id\/measurements  & El sistema devolverá un json con un mensaje de error por no encontrar la instancia correspondiente \\ \hline

  \end{longtable}
\end{center}

\begin{comment}
	Cuando paso un entero no positivo no responde con error sino que query or get 404 no encuentra el registro.
    El método get_latest_by_profile es del sprint2 cierto?
\end{comment}

\subsection{Pruebas ejecutadas}
A continuación se detalla la situación en la que quedaron las pruebas ejecutadas en los sprint anteriores, luego se desarrollarán las soluciones que se usaron y por última se cambiará el estado de aquellas errores encontrados por \textit{"cerrado"}.
        %
	\begin{itemize}
		\item \textbf{¿Qué fue bien?}
        	\begin{itemize}
				\item        Las cargas y muestra de mediciones en forma gráfica se llevan a cabo correctamente.
			\end{itemize}

   		\item \textbf{¿Qué se mejoró?}
        	\begin{itemize}
				\item \textbf{Cerrado} Al crear una nueva medición, se mostraba un cartel (alert de javascript) con una fecha, dicho alert fue eliminado.
                \item \textbf{Cerrado} Se encontró un problema con la zona horaria que usa el servidor y la zona horaria del usuario, para solucionarlo hubo q hacer un casteo previo cuando se solicitaba la fecha y hora del usuario para mostrar.
			\end{itemize}

   		\item \textbf{¿Qué se puede mejorar?}
        	\begin{itemize}
		        \item \textbf{Abierto} Solo debería mostrarse las unidades relacionadas al tipo de medición que se ha seleccionado 
				\item \textbf{Abierto} En el futuro se deberá mejorar las validaciones de los datos a la hora de cargar información en los formularios.
        		\item \textbf{Abierto} Se deberá mejorar la manera de seleccionar la fecha y la hora. 
                \item \textbf{Abierto} Deberá realizarse los carteles de advertencia necesarios.
            \end{itemize}
       
	\end{itemize}



\subsection{Estado final de pruebas}

	\begin{itemize}
		\item \textbf{¿Qué fue bien?}
        	\begin{itemize}
				\item        Las validaciones.
			\end{itemize}
            
		\item \textbf{¿Qué fue mal?}
        	\begin{itemize}
				\item        La librería que se utilizó al principio ``lvd3'' era de muy bajo nivel, por ello tuvo q cambiarse a D3 que es de más alto nivel. 
			\end{itemize}
            
   		\item \textbf{¿Qué se mejoró?}
        	\begin{itemize}
            	\item \textbf{Cerrado} Mal rendimiento de D3, mejorado usando código de la libreria lvd3
            	\item \textbf{Cerrado} Función de maximizar
                \item \textbf{Cerrado} Facilitar el pasaje del scope al controlador
                \item \textbf{Cerrado} Se encontró un problema con la zona horaria que usa el servidor y la zona horaria del usuario, para solucionarlo hubo q hacer un casteo previo cuando se solicitaba la fecha y hora del usuario para mostrar.
			\end{itemize}
       
	\end{itemize}