\section{Product Backlog}

Como se ha detallado, en el \textbf{Product Backlog} se recogen los requerimientos del sistema y necesidades de los clientes.
Sobre éstos, se realizará una estimación de tiempo necesario para concretarlo, se establecerán prioridades entre los \textit{User Stories}, y finalmente se indicará un comentario para el mismo, siempre que sea pertinente.

En la siguiente \textbf{Tabla} podemos apreciar el \textit{Product Backlog} del proyecto.
Se recuerda que los \textit{User Stories} incluidos representan una primera aproximación a los requerimientos definitivos, por lo que no puede considerarse que están establecidas todas y cada una de las características que el producto tendrá en su versión final.
Éstos serán actualizados y refinados constantemente durante el ciclo de vida del sistema.


{\scriptsize
\begin{tablaUSNumerada}
	\hline
        \multicolumn{1}{|c|}{\textbf{ID}} &
        \multicolumn{1}{|c|}{\textbf{Enunciado de la historia}} &
        \textbf{Prioridad} \\
	\hline
    \endhead
    
    \hline
        \label{infoPerfil} &
        Como paciente, quiero añadir información de mi perfil de salud o mediciones regulares, para que el médico cuente con más y mejor información al momento de realizar el diagnóstico. 
        & 10 de 10
        \\
    \hline
        \label{evitarPerdidas} &
        Como paciente, quiero añadir al sistema mis estudios realizados, para evitar posibles pérdidas. 
        & 9 de 10
        \\
    \hline
        \label{infoSalud} &
        Como paciente, quiero cargar mi información personal de salud referido a mediciones (altura, grasa corporal, peso, presión arterial), para que el médico cuente con más y mejor información al momento de realizar el diagnóstico. 
        & 10 de 10
        \\
    \hline
        \label{diagnosticarPaciente} &
        Como médico, quiero diagnosticar a un paciente, para darle un cierre a una incidencia planteada por la persona. 
        & 7 de 10
        \\
    \hline
        \label{cargaCentroSalud} &
        Como paciente, quiero que los sistemas de salud existentes puedan cargar sus resultados directamente en mi carpeta de salud, para centralizar mi información. 
        & 7 de 10
        \\
    \hline
        \label{asociarDispositivo} &
        Como paciente, quiero asociar un dispositivo, para agilizar y ampliar la carga de datos. 
        & 4 de 10
        \\
    \hline
        \label{categorizarEstudios} &
        Como paciente, quiero categorizar mis estudios por rama de medicina, para lograr una mejor organización y navegabilidad en el sistema. 
        & 7 de 10
        \\
    \hline
        \label{infoPaciente} &
        Como laboratorio, quiero cargar información de un paciente en su cuenta, para ahorrarle las molestias de volver. 
        & 7 de 10
        \\
    \hline
        \label{guardarInfoLocal} &
        Como paciente, quiero guardar mi información de manera local, para tener un respaldo. 
        & 8 de 10
        \\
    \hline
        \label{agregarGrupoFamiliar} &
        Como paciente, quiero agregar personas a mi grupo familiar, para llevar el seguimiento de los mismos. 
        & 8 de 10
        \\
    \hline
        \label{modificarPermisos} &
        Como paciente, quiero modificar los permisos de visualización de mis datos con respecto a cada uno de los integrantes del grupo familiar, para tener un control total sobre mi privacidad. 
        & 4 de 10
        \\
    \hline
        \label{comunicarResultado} &
        Como paciente, quiero que no sea necesario ir al hospital para que un médico me comunique los resultados del análisis.
        & 5 de 10
        \\
    \hline
        \label{registrarConFacebook} &
        Como usuario, quiero registrarme con una cuenta de Facebook y/o Google, para facilitar la inscripción al sitio y el manejo de credenciales. 
        & 4 de 10
        \\
    \hline
        \label{infoHijo} &
        Como mujer embarazada, quiero llevar la información de mi hijo, para transmitírsela cuando nazca. 
        & 2 de 10
        \\
    \hline
        \label{graficaParaMedico} &
        Como médico, quiero ver gráficas que resuman la información de un paciente, para poder ver sus cambios a lo largo de la historia y así apoyar la toma de decisiones y el diagnóstico. 
        & 6 de 10
        \\
    \hline
        \label{accesoCualquierLugar} &
        Como paciente, quiero acceder a mis documentos desde cualquier lugar, para hacer uso de ellos cuando los necesite. 
        & 5 de 10
        \\
    \hline
        \label{graficaParaPaciente} &
        Como paciente, quiero ver gráficas que resuman mi información en particular, para poder ver mis cambios a lo largo de la historia. 
        & 5 de 10
        \\
    \hline
        \label{resumenInfo} &
        Como paciente, quiero obtener un resumen de mi información de salud básica, para hacer uso de la misma en caso de una emergencia. 
        & 8 de 10
        \\
    \hline
        \label{mostrarComentario} &
        Como paciente, quiero ver en un único lugar los comentarios realizados por los médicos autorizados, para una lectura rápida. 
        & 8 de 10
        \\
    \hline
        \label{verificarPaciente} &
        Como médico, quiero verificar que las personas que solicitan mi atención sean pacientes, para mantener mi cantidad de consultas en un nivel controlable. 
        & 8 de 10
        \\
    \hline 
        \label{validarUsuario} &
        Como paciente, quiero contar con un acceso único y privado a mi información, para que no sea accedida por usuarios sin permisos.
        & 8 de 10
        \\
        \hline     
\end{tablaUSNumerada}
}